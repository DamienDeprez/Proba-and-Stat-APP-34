\documentclass[a4paper]{article}

\usepackage[utf8]{inputenc}
\usepackage[T1]{fontenc}      
\usepackage[english]{babel}
\usepackage{array}
% Layout and figures
\usepackage[top=2.5cm,bottom=2.5cm,right=2.5cm,left=2.5cm]{geometry}
\usepackage{subfigure}
\usepackage{rotating}
\usepackage{caption}
% Units and numbers
\usepackage[squaren, Gray]{SIunits}
\usepackage{sistyle}
\usepackage[autolanguage]{numprint}
% Math
\usepackage{amsmath}
\usepackage{amssymb}
\usepackage{amsthm}

% Sets
\newcommand{\Z}{\mathbb{Z}}
\newcommand{\R}{\mathbb{R}}
\newcommand{\C}{\mathbb{C}}
% Links
\usepackage{url}
\usepackage{hyperref}
\hypersetup{
    colorlinks,
    citecolor=black,
    filecolor=black,
    linkcolor=black,
    urlcolor=black
}
\usepackage{tikz}
\usetikzlibrary{arrows,automata}

\definecolor{mygreen}{rgb}{0,0.6,0}
\definecolor{mygray}{rgb}{0.5,0.5,0.5}
\definecolor{mymauve}{rgb}{0.58,0,0.82}

\usepackage{enumerate}
\usepackage{listings}
\lstset{ %
  backgroundcolor=\color{white},   % choose the background color; you must add \usepackage{color} or \usepackage{xcolor}
  basicstyle=\footnotesize,        % the size of the fonts that are used for the code
  breakatwhitespace=false,         % sets if automatic breaks should only happen at whitespace
  breaklines=true,                 % sets automatic line breaking
  captionpos=b,                    % sets the caption-position to bottom
  commentstyle=\color{mygreen},    % comment style
  deletekeywords={...},            % if you want to delete keywords from the given language
  escapeinside={\%*}{*)},          % if you want to add LaTeX within your code
  extendedchars=true,              % lets you use non-ASCII characters; for 8-bits encodings only, does not work with UTF-8
  frame=single,	                   % adds a frame around the code
  keepspaces=true,                 % keeps spaces in text, useful for keeping indentation of code (possibly needs columns=flexible)
  keywordstyle=\color{blue},       % keyword style
  otherkeywords={*,...},           % if you want to add more keywords to the set
  numbers=left,                    % where to put the line-numbers; possible values are (none, left, right)
  numbersep=5pt,                   % how far the line-numbers are from the code
  numberstyle=\tiny\color{mygray}, % the style that is used for the line-numbers
  rulecolor=\color{black},         % if not set, the frame-color may be changed on line-breaks within not-black text (e.g. comments (green here))
  showspaces=false,                % show spaces everywhere adding particular underscores; it overrides 'showstringspaces'
  showstringspaces=false,          % underline spaces within strings only
  showtabs=false,                  % show tabs within strings adding particular underscores
  stepnumber=2,                    % the step between two line-numbers. If it's 1, each line will be numbered
  stringstyle=\color{mymauve},     % string literal style
  tabsize=2,	                   % sets default tabsize to 2 spaces
  title=\lstname                   % show the filename of files included with \lstinputlisting; also try caption instead of title
}

%\usepackage{titlesec}
% New commands
\newcommand{\matlab}{\textsc{Matlab}}
\newcommand{\annexe}{\part{Annexes}\appendix}
\newcommand{\biblioreport}[1]{\bibliographystyle{plain}\bibliography{#1}\nocite{*}}
\DeclareMathOperator{\newdiff}{d} % use \dif instead
\newcommand{\dif}{\newdiff\!}
\newcommand{\fpart}[2]{\frac{\partial #1}{\partial #2}}
\newcommand{\ffpart}[2]{\frac{\partial^2 #1}{\partial #2^2}}
\newcommand{\fdpart}[3]{\frac{\partial^2 #1}{\partial #2\partial #3}}
\newcommand{\fdif}[2]{\frac{\dif #1}{\dif #2}}
\newcommand{\ffdif}[2]{\frac{\dif^2 #1}{\dif #2^2}}
\newcommand{\specialcell}[2][c]{%
\begin{tabular}[#1]{@{}c@{}}#2\end{tabular}}

%\titleformat{\section}
%{\bf\fontsize{20.74}{20.74}\selectfont}
%{\bf\fontsize{20.74}{20.74}\selectfont \thesection \hspace{1 cm}}
%{0 pt}{}{}
%\titlespacing{\section}{10 pt}{10 pt}{20 pt}[10 pt]

\begin{document}
\begin{titlepage}
\newcommand{\HRule}{\rule{\linewidth}{0.5mm}} 
\center 
\textsc{\Large Universit\'e Catholique de Louvain}\\[1cm] 
\textsc{\LARGE LFSAB1105 - Probability and Statistics}\\[0.5cm] 

\HRule \\[0.4cm]
{ \huge \bfseries APP}\\ [0.4cm]
\HRule \\[0.1cm]
\vspace{1cm}
%LOGO UCL
\begin{figure}[ht]
\centering
\includegraphics [height=10cm] {img/ucl}
\end{figure}
\vspace{1cm}
%Author Name
\begin{minipage}{0.7\textwidth}
\begin{center}
\begin{tabular}{lc}
Laurent \textsc{Deleu} & 0000-00-00 \\
Nicolas \textsc{Delinte} & 4801-13-00\\
Damien \textsc{Deprez} & 2893-13-00 \\
Bastien \textsc{Gillon} & 0000-00-00\\

\end{tabular}
\end{center}

\end{minipage}\\[1cm]

%----------------------------------------------------------------------------------------
%	DATE SECTION
%----------------------------------------------------------------------------------------

{\large Ann\'ee acad\'emique 2016-2017}\\[0,25cm] 
{\large \'Ecole Polytechnique de Louvain}\\[1cm]

%----------------------------------------------------------------------------------------
%	LOGO SECTION
%----------------------------------------------------------------------------------------

\begin{center}
  \includegraphics[width = 20mm]{img/epl.jpg} \hfill
\end{center}
%----------------------------------------------------------------------------------------

\vfill % Fill the rest of the page with whitespace
\end{titlepage}

\tableofcontents

\section{Descriptive statistics}

\begin{enumerate}[(a)]

\item In order to find $t_{p}$ the quantile of order p such that $ P(T \leq t_{p}) = p $, we need to find the solution to the following equation.

$$F_{\alpha , \beta}(t) = \int_{0}^{t_{p}} f_{\alpha , \beta}(t') dt' = p $$
\\
Before integrating the density function, note that :

$$ \frac{\mathrm{d} }{\mathrm{d} x} (1-t^\alpha)^\beta = - \alpha \beta t^{\alpha-1} (1-t^\alpha)^{\beta -1}$$
\\
Thus,

$$F_{\alpha , \beta} (t_p) \quad = \int_{0}^{t_{p}} \alpha \beta t'^{\alpha-1} (1-t'^\alpha)^{\beta -1} dt' = \left [  -(1-t'^\alpha)^\beta \right ]^{t_{p}}_{0} = 1-(1-t_{p}^{\alpha})^\beta$$
\\
The result can be then easily found : $\quad t_p = (1-(1-p)^{\frac{1}{\beta}})^{\frac{1}{\alpha}}$

\item Let's apply the general expression of the moment of order k to our r.v.

\nonumber
\begin{equation} \label{eq1}
\begin{split}
\mathbb{E}(T^k) & = \int_{0}^{1} t^k \alpha \beta t^{\alpha -1} (1-t^\alpha)^{\beta-1} dt \\
 & = \beta \int_{0}^{1} \tau^{\frac{k}{\alpha}} (1-\tau)^{\beta -1} d\tau \\
 & = \beta \frac{\Gamma(\frac{k}{\alpha}+1)\Gamma(\beta)}{\Gamma(\frac{k}{\alpha}+1+\beta)}
\end{split}
\end{equation}
\\
In order to obtain an expression similar to the hint given in the instructions, it necessary to use a variable change : $\tau = t^\alpha$.

\item The previous results allow us to find some statistics describing our variable T.
\\
$\textbf{Mean :} \quad \mu = \mathbb{E}(T) = \beta \frac{\Gamma(\frac{1}{\alpha}+1)\Gamma(\beta)}{\Gamma(\frac{1}{\alpha}+1+\beta)} $
\\
$\textbf{Variance :} \quad \sigma^2 = \mathbb{E}(T^2)-(\mathbb{E}(T))^2 = \beta\frac{\Gamma(\frac{2}{\alpha}+1)\Gamma(\beta)}{\Gamma(\frac{2}{\alpha}+1+\beta)} - \beta^2 \frac{\Gamma(\frac{1}{\alpha}+1)^2\Gamma(\beta)^2}{\Gamma(\frac{1}{\alpha}+1+\beta)^2}$
\\
$\textbf{Median :} \quad t_{0.5} = (1-\frac{1}{2^{\frac{1}{\beta}}})^\frac{1}{\alpha}$
\\
$\textbf{Range :} \quad \textup{Range} = 1$ since $f$ is strictly positive $\forall t \in \left ]  0;1 \right [ $
\\
$\textbf{Interquartile range :} \quad \textup{IQR} = t_{0.75} - t_{0.25} = (1-\frac{1}{4^\frac{1}{\beta}})^\frac{1}{\alpha} - (1-(\frac{3}{4})^\frac{1}{\beta})^\frac{1}{\alpha}$
\\
$\textbf{Coefficient of variation :} \quad cv = \frac{\sigma}{\mu} = \frac{\sqrt{\beta\frac{\Gamma(\frac{2}{\alpha}+1)\Gamma(\beta)}{\Gamma(\frac{2}{\alpha}+1+\beta)} - \beta^2 \frac{\Gamma(\frac{1}{\alpha}+1)^2\Gamma(\beta)^2}{\Gamma(\frac{1}{\alpha}+1+\beta)^2}}}{\beta \frac{\Gamma(\frac{1}{\alpha}+1)\Gamma(\beta)}{\Gamma(\frac{1}{\alpha}+1+\beta)}}$

\item Let's fix our parameters to have an idea of the values of our statistics. For $\alpha_0 = 2$ and $\beta_0 = 4$, we obtain : $\mu = 0.4063$ , $\sigma^2 = 0.0349$ , $t_{0.5} = 0.3989$ , $\textup{IQR} = 0.2778$ and $cv = 0.4596$.

\end{enumerate}

\section{Estimation}

\section{Simulations}

\section{Linear regression and ANOVA}

\appendix

\section{MatLab Code}
\end{document}
